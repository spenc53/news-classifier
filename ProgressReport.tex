\documentclass[fleqn,11pt]{article}

\usepackage[letterpaper,margin=0.75in]{geometry}
\usepackage{amsmath}
\usepackage{booktabs}
\usepackage{graphicx}
\usepackage{listings}
\usepackage{mathtools}
\usepackage{fixltx2e}
\usepackage{hyperref}
\usepackage{color}

\setlength{\parindent}{1.4em}

\begin{document}

\lstset{
  language=Python,
  basicstyle=\small,          % print whole listing small
  keywordstyle=\bfseries,
  identifierstyle=,           % nothing happens
  commentstyle=,              % white comments
  stringstyle=\ttfamily,      % typewriter type for strings
  showstringspaces=false,     % no special string spaces
  numbers=left,
  numberstyle=\tiny,
  numbersep=5pt,
  frame=tb,
}

\title{Progress Report}
\author{Wilson Fearn, Andrew Hale, Victor Lazaro, Spencer Seeger}
\date{March 13, 2018}
\maketitle

\begin{itemize}
  \item A description of the problem
    \begin{itemize}
      \item The problem that we are solving is allowing machine learning algorithms to take news articles and identify the topics of them.
    \end{itemize}
  \item What machine learning model they are initially trying to learn with
    \begin{itemize}
      \item The machine learning algorithms we are planning on using include Na{\"i}ve Bayes, Subject Vector Machines, and Grid Search.
    \end{itemize}
  \item How and from where are they gathering data
    \begin{itemize}
      \item We will be gathering data from many different news websites, including www.washingtonpost.com and www.cnn.com.
    \end{itemize}
  \item A description of their data set including:
    \begin{itemize}
      \item Actual example instances, including a reasonable representation (continuous, nominal, etc.) and values for each feature
        \begin{itemize}
          \item \textcolor{red}{FILL THIS IN}
        \end{itemize}
      \item How many instances and features you plan to have in your final data set
        \begin{itemize}
          \item \textcolor{red}{FILL THIS IN}
        \end{itemize}
    \end{itemize}
    \item Brief discussion of plans and schedule to finish the project
    \begin{itemize}
      \item We plan to implement the different machine learning models as quickly as possible, and then submit the project before April 3, 2018.
    \end{itemize}
\end{itemize}

\end{document}